%\documentclass[11pt,reqno]{amsart}
\documentclass[11pt,reqno]{article}
\usepackage[margin=.8in, paperwidth=8.5in, paperheight=11in]{geometry}
%\usepackage{geometry}                % See geometry.pdf to learn the layout options. There are lots.
%\geometry{letterpaper}                   % ... or a4paper or a5paper or ... 
%\geometry{landscape}                % Activate for for rotated page geometry
%\usepackage[parfill]{parskip}    % Activate to begin paragraphs with an empty line rather than an indent7
\usepackage{graphicx}
\usepackage{pstricks}
\usepackage{amssymb}
\usepackage{epstopdf}
\usepackage{amsmath}
\usepackage{subfigure}
\usepackage{caption}
\pagestyle{plain}
%\renewcommand{\topfraction}{0.3}
%\renewcommand{\bottomfraction}{0.8}
%\renewcommand{\textfraction}{0.07}
\DeclareGraphicsRule{.tif}{png}{.png}{`convert #1 `dirname #1`/`basename #1 .tif`.png}

\title{Probability and Random Variables: \\ Assignment 3}
\author{Andrew Rickert}
\date{Started: February ??, 2012 \\ \hspace{1pt} Ended: February ??,  2012}                                           % Activate to display a given date or no date

\begin{document}
\maketitle

\noindent \framebox[1.1\width]{\textbf{Part A}} \par

% Page 1
\begin{flushleft} 
\textbf{Class 18.440} - Chapter 3 Problem 24\\
\rule{500pt}{1pt}\\
\end{flushleft} 


\vspace{15pt}
\begin{flushleft} 
\textbf{Class 18.440} - Chapter 3 Problem 38\\
\rule{500pt}{1pt}\\
\end{flushleft} 


\vspace{15pt}
\begin{flushleft} 
\textbf{Class 18.440} - Chapter 3 Problem 43\\
\rule{500pt}{1pt}\\
\end{flushleft} 


\vspace{15pt}
\begin{flushleft} 
\textbf{Class 18.440} - Chapter 3 Problem 76\\
\rule{500pt}{1pt}\\
\end{flushleft} 
  
  
\vspace{15pt}
\begin{flushleft} 
\textbf{Class 18.440} - Chapter 3 Theoretical Exercise 24\\
\rule{500pt}{1pt}\\
\end{flushleft} 


\vspace{15pt}
\begin{flushleft} 
\textbf{Class 18.440} - Chapter 3 Self Test Problem/Exercise 8\\
\rule{500pt}{1pt}\\
\end{flushleft} 


\vspace{15pt}
\begin{flushleft} 
\textbf{Class 18.440} - Chapter 3 Self Test Problem/Exercise 14\\
\rule{500pt}{1pt}\\
\end{flushleft} 


\vspace{15pt}
\begin{flushleft} 
\textbf{Class 18.440} - Chapter 3 Self Test Problem/Exercise 22\\
\rule{500pt}{1pt}\\
\end{flushleft} 


\noindent \framebox[1.1\width]{\textbf{Part B}} \par

Suppose that a fair coin is tossed infinitely many times, independently. Let $X_i$ denote the outcome of the ith coin toss (an element of \{H, T\}). Compute the probability that:

\noindent1. $X_i = H$ for all positive integers $i$.\\
2. The pattern $HHTTHHTT$ occurs at some point in the sequence $X1,X2,X3,\ldots$ \\

\noindent \framebox[1.1\width]{\textbf{Part C}} \par

Two unfair dice are tossed. Let $p_{i,j}$, for $i$ and $j$ in $\{1, 2, 3, 4, 5, 6\}$, denote the probability that the first die comes up $i$ and the second $j$. Suppose that for any $i$ and $j$ in $\{1, 2, 3, 4, 5, 6\}$ the event that the first die comes up $i$ is independent of the event that the second die comes up $j$. Show that this independence implies that, as a 6 by 6 matrix, $p_{i,j}$ has rank one (i.e., show that there is some column of the matrix such that each of the other five column vectors is a constant multiple of that one).

\end{document}  